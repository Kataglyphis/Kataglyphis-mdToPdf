% Awesome Source CV LaTeX Template
%
% This template has been downloaded from:
% https://github.com/darwiin/awesome-neue-latex-cv
%
% Author:
% Christophe Roger
%
% Template license:
% CC BY-SA 4.0 (https://creativecommons.org/licenses/by-sa/4.0/)
%Section: Project
\IfLanguageName{english}{
    \sectionTitle{Projects}{\faLaptop}

    \begin{projects}

        \project
    	{Renderer}{2020-now}
    	{\github{Kataglyphis/GraphicsEngineVulkan} \github{Kataglyphis/GraphicEngine}}
        {This projects provide me with a solid Vulkan/OpenGL starting point for implementing modern established rendering techniques and getting quickly started in own research topics.}
    	{CMake, C, C++, Python, Vulkan, OpenGL}
    				
    	\project
    	{Machine Learning Algorithms}{2020 - now}
    	{\github{Kataglyphis/MachineLearningAlgorithms}}
    	{This project provides me a solid Machine Learning(ML) starting point for implementing modern established ML algorithms and getting quickly started in own research topics.}
    	{\LaTeX, Python, R}
     
        %\project
    	%{Designing User-adaptive Content for Mixed Reality Using Eye and Hand Tracking}{2022-2023}
    	%{\github{Kataglyphis/Designing-User-adaptive-Content-for-Mixed-Reality-Using-Eye-and-Hand-Tracking}}
    	%{This work exhibits interrelationships between users' currently fixated objects, their eye, head, hand movement and users' intentions while doing a task. Based on these findings this work provides user-adaptive guidance for improved interaction quality between Mixed Reality environment and the user.}
    	%{\LaTeX, C\sharp, Mixed Reality, Python, Unity}


    	%\project
    	%{Temporary Stable Blue Noise Error Distribution in Screen Space for Real Time Applications}{2019 - 2020}
    	%{\github{Kataglyphis/BachelorArbeit}}
    	%{Tracing Rays and familiar techniques are currently gaining importance. Former papers had shown the effectiveness of blue-noise error distributions and %their importance in increasing optical image quality. My work is building upon this findings and presents a temporary algorithm to find a blue noise %error distribution in screen space.}
    	%{\LaTeX, C, C++, Python}

        
        %\project
    	%{GPU collision detection with acceleration structures}{2021-2022}
    	%{\github{Kataglyphis/GPUCollisionDetectionWithAS}}
    	%{New form of collision detection using acceleration structures. Implementation of %a particle system accelerated on the GPU}
    	%{\LaTeX,C, C++, Vulkan API, GPGPU}
    	
    \end{projects}
}{
    \sectionTitle{Projekte}{\faLaptop}

    \begin{projects}
        \project
    	{Renderer}{2020-now}
    	{\github{Kataglyphis/GraphicsEngineVulkan}\github{Kataglyphis/GraphicEngine}}
    	{Diese Projekte bieten mir einen soliden Startpunkt um diverse etablierte Computergrafik Algorithmen nachzuimplementieren und darauf aufbauend auch eigene Forschung zu betreiben.}
    	{\LaTeX, CMake, C, C++, Python, Vulkan, OpenGL}
        
    	\project
    	{Machine Learning Algorithms}{2020 - now}
    	{\github{Kataglyphis/MachineLearningAlgorithms}}
    	{Dieses Projekt bietet mir einen soliden Startpunkt um diverse etablierte Machine Learning(ML) Algorithmen nachzuimplementieren und darauf aufbauend auch eigene Forschung zu betreiben.}
    	{\LaTeX, Python, R}
     
    	%\project
    	%{Zeitlich stabile Blue Noise Fehlerverteilung im Bildraum für Echtzeitanwendungen}{2019 - 2020}
    	%{\github{Kataglyphis/BachelorArbeit}}
    	%{Effektivität von Blue Noise Fehlerverteilungen zur Steigerung der Bildqualität.}
    	%{\LaTeX,C, C++, Python}
    	
        %\project
    	%{GPU collision detection with acceleration structures}{2021-2022}
    	%{\github{Kataglyphis/GPUCollisionDetectionWithAS}}
    	%{Neue Form der Kollisionsdetektion mit Hile von Beschleunigungsstrukturen. %Implementierung eines Partikelsystems, dass durch die GPU beschleunigt wird.}
    	%{\LaTeX,C, C++, Vulkan API, GPGPU}

        %\project
    	%{Entwurf von benutzeranpassbarem Szeneninhalt für Mixed Reality mit Hilfe von Augen- und Handnachverfolgung}{2022-2023}
    	%{\github{Kataglyphis/Designing-User-adaptive-Content-for-Mixed-Reality-Using-Eye-and-Hand-Tracking}}
    	%{Diese Arbeit zeigt Wechselbeziehungen zwischen den aktuell fixierten Objekten der Benutzer, ihren Augen-, Kopf- und Handbewegungen und den Absichten der Benutzer bei der Ausführung einer Aufgabe. Basierend auf diesen Erkenntnissen bietet diese Arbeit benutzeradaptive Führung für eine verbesserte Interaktionsqualität zwischen der Mixed Reality-Umgebung und dem Benutzer.}
    	%{\LaTeX, C\sharp, Mixed Reality, Python, Unity}
        
    \end{projects}
}
