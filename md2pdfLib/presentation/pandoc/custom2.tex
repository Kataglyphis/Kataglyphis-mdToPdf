% Options for packages loaded elsewhere
\PassOptionsToPackage{unicode$for(hyperrefoptions)$,$hyperrefoptions$$endfor$}{hyperref}
\PassOptionsToPackage{hyphens}{url}
$if(colorlinks)$
\PassOptionsToPackage{dvipsnames,svgnames,x11names}{xcolor}
$endif$
$if(CJKmainfont)$
\PassOptionsToPackage{space}{xeCJK}
$endif$

\documentclass[
$if(fontsize)$
  $fontsize$,
$endif$
$if(papersize)$
  $papersize$paper,
$endif$
$if(beamer)$
  % ignorenonframetext,
$if(handout)$
  handout,
$endif$
$if(aspectratio)$
  aspectratio=$aspectratio$,
$endif$
$endif$
$for(classoption)$
  $classoption$$sep$,
$endfor$
]{$documentclass$}

$if(beamer)$
$if(background-image)$
\usebackgroundtemplate{%
  \includegraphics[width=\paperwidth]{$background-image$}%
}
% In beamer background-image does not work well when other images are used, so this is the workaround
\pgfdeclareimage[width=\paperwidth,height=\paperheight]{background}{$background-image$}
\usebackgroundtemplate{\pgfuseimage{background}}
$endif$
\usepackage{pgfpages}
\setbeamertemplate{caption}[numbered]
\setbeamertemplate{caption label separator}{: }
\setbeamercolor{caption name}{fg=normal text.fg}
\beamertemplatenavigationsymbols$if(navigation)$$navigation$$else$empty$endif$
$for(beameroption)$
\setbeameroption{$beameroption$}
$endfor$
% Prevent slide breaks in the middle of a paragraph
\widowpenalties 1 10000
\raggedbottom
$if(section-titles)$
\setbeamertemplate{part page}{
  \centering
  \begin{beamercolorbox}[sep=16pt,center]{part title}
    \usebeamerfont{part title}\insertpart\par
  \end{beamercolorbox}
}
\setbeamertemplate{section page}{
  \centering
  \begin{beamercolorbox}[sep=12pt,center]{part title}
    \usebeamerfont{section title}\insertsection\par
  \end{beamercolorbox}
}
\setbeamertemplate{subsection page}{
  \centering
  \begin{beamercolorbox}[sep=8pt,center]{part title}
    \usebeamerfont{subsection title}\insertsubsection\par
  \end{beamercolorbox}
}
\AtBeginPart{
  \frame{\partpage}
}
\AtBeginSection{
  \ifbibliography
  \else
    \frame{\sectionpage}
  \fi
}
\AtBeginSubsection{
  \frame{\subsectionpage}
}
$endif$
$endif$
$if(beamerarticle)$
\usepackage{beamerarticle} % needs to be loaded first
$endif$
\usepackage{amsmath,amssymb}
$if(linestretch)$
\usepackage{setspace}
$endif$
\usepackage{iftex}
\ifPDFTeX
  \usepackage[$if(fontenc)$$fontenc$$else$T1$endif$]{fontenc}
  \usepackage[utf8]{inputenc}
  \usepackage{textcomp} % provide euro and other symbols
\else % if luatex or xetex
$if(mathspec)$
  \ifXeTeX
    \usepackage{mathspec} % this also loads fontspec
  \else
    \usepackage{unicode-math} % this also loads fontspec
  \fi
$else$
  \usepackage{unicode-math} % this also loads fontspec
$endif$
  \defaultfontfeatures{Scale=MatchLowercase}$-- must come before Beamer theme
  \defaultfontfeatures[\rmfamily]{Ligatures=TeX,Scale=1}
\fi
$if(fontfamily)$
$else$
$-- Set default font before Beamer theme so the theme can override it
\usepackage{lmodern}
$endif$

$-- Set Beamer theme before user font settings so they can override theme
$if(beamer)$
$if(theme)$
\usetheme[$for(themeoptions)$$themeoptions$$sep$,$endfor$]{$theme$}
$endif$
$if(colortheme)$
\usecolortheme{$colortheme$}
$endif$
$if(fonttheme)$
\usefonttheme{$fonttheme$}
$endif$
$if(mainfont)$
\usefonttheme{serif} % use mainfont rather than sansfont for slide text
$endif$
$if(innertheme)$
\useinnertheme{$innertheme$}
$endif$
$if(outertheme)$
\useoutertheme{$outertheme$}
$endif$
$endif$

$-- User font settings (must come after default font and Beamer theme)
$if(fontfamily)$
\usepackage[$for(fontfamilyoptions)$$fontfamilyoptions$$sep$,$endfor$]{$fontfamily$}
$endif$
\ifPDFTeX\else
  % xetex/luatex font selection
$if(mainfont)$
  $if(mainfontfallback)$
    \ifLuaTeX
      \usepackage{luaotfload}
      \directlua{luaotfload.add_fallback("mainfontfallback",{
        $for(mainfontfallback)$"$mainfontfallback$"$sep$,$endfor$
      })}
    \fi
  $endif$
  \setmainfont[$for(mainfontoptions)$$mainfontoptions$$sep$,$endfor$$if(mainfontfallback)$,RawFeature={fallback=mainfontfallback}$endif$]{$mainfont$}
$endif$
$if(sansfont)$
  $if(sansfontfallback)$
    \ifLuaTeX
      \usepackage{luaotfload}
      \directlua{luaotfload.add_fallback("sansfontfallback",{
        $for(sansfontfallback)$"$sansfontfallback$"$sep$,$endfor$
      })}
    \fi
  $endif$
  \setsansfont[$for(sansfontoptions)$$sansfontoptions$$sep$,$endfor$$if(sansfontfallback)$,RawFeature={fallback=sansfontfallback}$endif$]{$sansfont$}
$endif$

$if(monofont)$
  $if(monofontfallback)$
    \ifLuaTeX
      \usepackage{luaotfload}
      \directlua{luaotfload.add_fallback("monofontfallback",{
        $for(monofontfallback)$"$monofontfallback$"$sep$,$endfor$
      })}
    \fi
  $endif$
  \setmonofont[$for(monofontoptions)$$monofontoptions$$sep$,$endfor$$if(monofontfallback)$,RawFeature={fallback=monofontfallback}$endif$]{$monofont$}
$endif$
$for(fontfamilies)$
  \newfontfamily{$fontfamilies.name$}[$for(fontfamilies.options)$$fontfamilies.options$$sep$,$endfor$]{$fontfamilies.font$}
$endfor$
$if(mathfont)$
$if(mathspec)$
  \ifXeTeX
    \setmathfont(Digits,Latin,Greek)[$for(mathfontoptions)$$mathfontoptions$$sep$,$endfor$]{$mathfont$}
  \else
    \setmathfont[$for(mathfontoptions)$$mathfontoptions$$sep$,$endfor$]{$mathfont$}
  \fi
$else$
  \setmathfont[$for(mathfontoptions)$$mathfontoptions$$sep$,$endfor$]{$mathfont$}
$endif$
$endif$

$if(CJKmainfont)$
  \ifXeTeX
    \usepackage{xeCJK}
    \setCJKmainfont[$for(CJKoptions)$$CJKoptions$$sep$,$endfor$]{$CJKmainfont$}
    $if(CJKsansfont)$
      \setCJKsansfont[$for(CJKoptions)$$CJKoptions$$sep$,$endfor$]{$CJKsansfont$}
    $endif$
    $if(CJKmonofont)$
      \setCJKmonofont[$for(CJKoptions)$$CJKoptions$$sep$,$endfor$]{$CJKmonofont$}
    $endif$
  \fi
$endif$
$if(luatexjapresetoptions)$
  \ifLuaTeX
    \usepackage[$for(luatexjapresetoptions)$$luatexjapresetoptions$$sep$,$endfor$]{luatexja-preset}
  \fi
$endif$
$if(CJKmainfont)$
  \ifLuaTeX
    \usepackage[$for(luatexjafontspecoptions)$$luatexjafontspecoptions$$sep$,$endfor$]{luatexja-fontspec}
    \setmainjfont[$for(CJKoptions)$$CJKoptions$$sep$,$endfor$]{$CJKmainfont$}
  \fi
$endif$
\fi
$if(zero-width-non-joiner)$%% Support for zero-width non-joiner characters.
\makeatletter
\def\zerowidthnonjoiner{%
  % Prevent ligatures and adjust kerning, but still support hyphenating.
  \texorpdfstring{%
    \TextOrMath{\nobreak\discretionary{-}{}{\kern.03em}%
      \ifvmode\else\nobreak\hskip\z@skip\fi}{}%
  }{}%
}
\makeatother
\ifPDFTeX
  \DeclareUnicodeCharacter{200C}{\zerowidthnonjoiner}
\else
  \catcode`^^^^200c=\active
  \protected\def ^^^^200c{\zerowidthnonjoiner}
\fi
%% End of ZWNJ support
$endif$

\IfFileExists{upquote.sty}{\usepackage{upquote}}{}
\IfFileExists{microtype.sty}{% use microtype if available
  \usepackage[$for(microtypeoptions)$$microtypeoptions$$sep$,$endfor$]{microtype}
  \UseMicrotypeSet[protrusion]{basicmath} % disable protrusion for tt fonts
}{}
$if(indent)$
$else$
\makeatletter
\@ifundefined{KOMAClassName}{% if non-KOMA class
  \IfFileExists{parskip.sty}{%
    \usepackage{parskip}
  }{% else
    \setlength{\parindent}{0pt}
    \setlength{\parskip}{6pt plus 2pt minus 1pt}}
}{% if KOMA class
  \KOMAoptions{parskip=half}}
\makeatother
$endif$
$if(verbatim-in-note)$
\usepackage{fancyvrb}
$endif$
\usepackage{xcolor}
$if(geometry)$
$if(beamer)$
\geometry{$for(geometry)$$geometry$$sep$,$endfor$}
$else$
\usepackage[$for(geometry)$$geometry$$sep$,$endfor$]{geometry}
$endif$
$endif$
$if(beamer)$
\newif\ifbibliography
$endif$
$if(listings)$
\usepackage{listings}
\newcommand{\passthrough}[1]{#1}
\lstset{defaultdialect=[5.3]Lua}
\lstset{defaultdialect=[x86masm]Assembler}
$endif$
$if(lhs)$
\lstnewenvironment{code}{\lstset{language=Haskell,basicstyle=\small\ttfamily}}{}
$endif$
$if(highlighting-macros)$
$highlighting-macros$
$endif$
$if(tables)$
\usepackage{longtable,booktabs,array}
$if(multirow)$
\usepackage{multirow}
$endif$
\usepackage{calc} % for calculating minipage widths
$if(beamer)$
\usepackage{caption}
% Make caption package work with longtable
\makeatletter
\def\fnum@table{\tablename~\thetable}
\makeatother
$else$
% Correct order of tables after \paragraph or \subparagraph
\usepackage{etoolbox}
\makeatletter
\patchcmd\longtable{\par}{\if@noskipsec\mbox{}\fi\par}{}{}
\makeatother
% Allow footnotes in longtable head/foot
\IfFileExists{footnotehyper.sty}{\usepackage{footnotehyper}}{\usepackage{footnote}}
\makesavenoteenv{longtable}
$endif$
$endif$
$if(graphics)$
\usepackage{graphicx}
\makeatletter
\def\maxwidth{\ifdim\Gin@nat@width>\linewidth\linewidth\else\Gin@nat@width\fi}
\def\maxheight{\ifdim\Gin@nat@height>\textheight\textheight\else\Gin@nat@height\fi}
\makeatother
% Scale images if necessary, so that they will not overflow the page
% margins by default, and it is still possible to overwrite the defaults
% using explicit options in \includegraphics[width, height, ...]{}
\setkeys{Gin}{width=\maxwidth,height=\maxheight,keepaspectratio}
% Set default figure placement to htbp
\makeatletter
\def\fps@figure{htbp}
\makeatother
$endif$
$if(svg)$
\usepackage{svg}
$endif$
$if(strikeout)$
$-- also used for underline
\ifLuaTeX
  \usepackage{luacolor}
  \usepackage[soul]{lua-ul}
\else
  \usepackage{soul}
$if(beamer)$
  \makeatletter
  \let\HL\hl
  \renewcommand\hl{% fix for beamer highlighting
    \let\set@color\beamerorig@set@color
    \let\reset@color\beamerorig@reset@color
    \HL}
  \makeatother
$endif$
$if(CJKmainfont)$
  \ifXeTeX
    % soul's \st doesn't work for CJK:
    \usepackage{xeCJKfntef}
    \renewcommand{\st}[1]{\sout{#1}}
  \fi
$endif$
\fi
$endif$
$if(tables)$
\usepackage{longtable,booktabs,array}
$if(multirow)$
\usepackage{multirow}
$endif$
\usepackage{calc} % for calculating minipage widths
$if(beamer)$
\usepackage{caption}
% Make caption package work with longtable
\makeatletter
\def\fnum@table{\tablename~\thetable}
\makeatother
$else$
% Correct order of tables after \paragraph or \subparagraph
\usepackage{etoolbox}
\makeatletter
\patchcmd\longtable{\par}{\if@noskipsec\mbox{}\fi\par}{}{}
\makeatother
% Allow footnotes in longtable head/foot
\IfFileExists{footnotehyper.sty}{\usepackage{footnotehyper}}{\usepackage{footnote}}
\makesavenoteenv{longtable}
$endif$
$endif$
$if(graphics)$
\usepackage{graphicx}
\makeatletter
\def\maxwidth{\ifdim\Gin@nat@width>\linewidth\linewidth\else\Gin@nat@width\fi}
\def\maxheight{\ifdim\Gin@nat@height>\textheight\textheight\else\Gin@nat@height\fi}
\makeatother
% Scale images if necessary, so that they will not overflow the page
% margins by default, and it is still possible to overwrite the defaults
% using explicit options in \includegraphics[width, height, ...]{}
\setkeys{Gin}{width=\maxwidth,height=\maxheight,keepaspectratio}
% Set default figure placement to htbp
\makeatletter
\def\fps@figure{htbp}
\makeatother
$endif$
$if(svg)$
\usepackage{svg}
$endif$
$if(strikeout)$
$-- also used for underline
\ifLuaTeX
  \usepackage{luacolor}
  \usepackage[soul]{lua-ul}
\else
  \usepackage{soul}
$if(beamer)$
  \makeatletter
  \let\HL\hl
  \renewcommand\hl{% fix for beamer highlighting
    \let\set@color\beamerorig@set@color
    \let\reset@color\beamerorig@reset@color
    \HL}
  \makeatother
$endif$
$if(CJKmainfont)$
  \ifXeTeX
    % soul's \st doesn't work for CJK:
    \usepackage{xeCJKfntef}
    \renewcommand{\st}[1]{\sout{#1}}
  \fi
$endif$
\fi
$endif$
\setlength{\emergencystretch}{3em} % prevent overfull lines
\providecommand{\tightlist}{%
  \setlength{\itemsep}{0pt}\setlength{\parskip}{0pt}}
$if(numbersections)$
\setcounter{secnumdepth}{$if(secnumdepth)$$secnumdepth$$else$5$endif$}
$else$
\setcounter{secnumdepth}{-\maxdimen} % remove section numbering
$endif$
$if(subfigure)$
\usepackage{subcaption}
$endif$
$if(beamer)$
$else$
$if(block-headings)$
% Make \paragraph and \subparagraph free-standing
\makeatletter
\ifx\paragraph\undefined\else
  \let\oldparagraph\paragraph
  \renewcommand{\paragraph}{
    \@ifstar
      \xxxParagraphStar
      \xxxParagraphNoStar
  }
  \newcommand{\xxxParagraphStar}[1]{\oldparagraph*{#1}\mbox{}}
  \newcommand{\xxxParagraphNoStar}[1]{\oldparagraph{#1}\mbox{}}
\fi
\ifx\subparagraph\undefined\else
  \let\oldsubparagraph\subparagraph
  \renewcommand{\subparagraph}{
    \@ifstar
      \xxxSubParagraphStar
      \xxxSubParagraphNoStar
  }
  \newcommand{\xxxSubParagraphStar}[1]{\oldsubparagraph*{#1}\mbox{}}
  \newcommand{\xxxSubParagraphNoStar}[1]{\oldsubparagraph{#1}\mbox{}}
\fi
\makeatother
$endif$
$endif$$if(pagestyle)$
\pagestyle{$pagestyle$}
$endif$
$if(csl-refs)$
% definitions for citeproc citations
\NewDocumentCommand\citeproctext{}{}
\NewDocumentCommand\citeproc{mm}{%
  \begingroup\def\citeproctext{#2}\cite{#1}\endgroup}
\makeatletter
 % allow citations to break across lines
 \let\@cite@ofmt\@firstofone
 % avoid brackets around text for \cite:
 \def\@biblabel#1{}
 \def\@cite#1#2{{#1\if@tempswa , #2\fi}}
\makeatother
\newlength{\cslhangindent}
\setlength{\cslhangindent}{1.5em}
\newlength{\csllabelwidth}
\setlength{\csllabelwidth}{3em}
\newenvironment{CSLReferences}[2] % #1 hanging-indent, #2 entry-spacing
 {\begin{list}{}{%
  \setlength{\itemindent}{0pt}
  \setlength{\leftmargin}{0pt}
  \setlength{\parsep}{0pt}
  % turn on hanging indent if param 1 is 1
  \ifodd #1
   \setlength{\leftmargin}{\cslhangindent}
   \setlength{\itemindent}{-1\cslhangindent}
  \fi
  % set entry spacing
  \setlength{\itemsep}{#2\baselineskip}}}
 {\end{list}}
\usepackage{calc}
\newcommand{\CSLBlock}[1]{\hfill\break\parbox[t]{\linewidth}{\strut\ignorespaces#1\strut}}
\newcommand{\CSLLeftMargin}[1]{\parbox[t]{\csllabelwidth}{\strut#1\strut}}
\newcommand{\CSLRightInline}[1]{\parbox[t]{\linewidth - \csllabelwidth}{\strut#1\strut}}
\newcommand{\CSLIndent}[1]{\hspace{\cslhangindent}#1}
$endif$

$if(lang)$
\ifLuaTeX
\usepackage[bidi=basic]{babel}
\else
\usepackage[bidi=default]{babel}
\fi
$if(babel-lang)$
\babelprovide[main,import]{$babel-lang$}
$if(mainfont)$
\ifPDFTeX
\else
\babelfont{rm}[$for(mainfontoptions)$$mainfontoptions$$sep$,$endfor$$if(mainfontfallback)$,RawFeature={fallback=mainfontfallback}$endif$]{$mainfont$}
\fi
$endif$
$endif$
$for(babel-otherlangs)$
\babelprovide[import]{$babel-otherlangs$}
$endfor$
$for(babelfonts/pairs)$
\babelfont[$babelfonts.key$]{rm}{$babelfonts.value$}
$endfor$
% get rid of language-specific shorthands (see #6817):
\let\LanguageShortHands\languageshorthands
\def\languageshorthands#1{}
$endif$
$for(header-includes)$
$header-includes$
$endfor$
\ifLuaTeX
  \usepackage{selnolig}  % disable illegal ligatures
\fi
$if(dir)$
\ifPDFTeX
  \TeXXeTstate=1
  \newcommand{\RL}[1]{\beginR #1\endR}
  \newcommand{\LR}[1]{\beginL #1\endL}
  \newenvironment{RTL}{\beginR}{\endR}
  \newenvironment{LTR}{\beginL}{\endL}
\fi
$endif$
$if(natbib)$
\usepackage[$natbiboptions$]{natbib}
\bibliographystyle{$if(biblio-style)$$biblio-style$$else$plainnat$endif$}
$endif$
$if(biblatex)$
\usepackage[$if(biblio-style)$style=$biblio-style$,$endif$$for(biblatexoptions)$$biblatexoptions$$sep$,$endfor$]{biblatex}
$for(bibliography)$
\addbibresource{$bibliography$}
$endfor$
$endif$

$if(nocite-ids)$
\nocite{$for(nocite-ids)$$it$$sep$, $endfor$}
$endif$
$if(csquotes)$
\usepackage{csquotes}
$endif$
\usepackage{bookmark}
\IfFileExists{xurl.sty}{\usepackage{xurl}}{} % add URL line breaks if available
\urlstyle{$if(urlstyle)$$urlstyle$$else$same$endif$}
$if(links-as-notes)$
% Make links footnotes instead of hotlinks:
\DeclareRobustCommand{\href}[2]{#2\footnote{\url{#1}}}
$endif$
$if(verbatim-in-note)$
\VerbatimFootnotes % allow verbatim text in footnotes
$endif$
\hypersetup{
$if(title-meta)$
  pdftitle={$title-meta$},
$endif$
$if(author-meta)$
  pdfauthor={$author-meta$},
$endif$
$if(lang)$
  pdflang={$lang$},
$endif$
$if(subject)$
  pdfsubject={$subject$},
$endif$
$if(keywords)$
  pdfkeywords={$for(keywords)$$keywords$$sep$, $endfor$},
$endif$
$if(colorlinks)$
  colorlinks=true,
  linkcolor={$if(linkcolor)$$linkcolor$$else$Maroon$endif$},
  filecolor={$if(filecolor)$$filecolor$$else$Maroon$endif$},
  citecolor={$if(citecolor)$$citecolor$$else$Blue$endif$},
  urlcolor={$if(urlcolor)$$urlcolor$$else$Blue$endif$},
$else$
$if(boxlinks)$
$else$
  hidelinks,
$endif$
$endif$
  pdfcreator={LaTeX via pandoc}}

  $if(title)$
\title{$title$$if(thanks)$\thanks{$thanks$}$endif$}
$endif$
$if(subtitle)$
$if(beamer)$
$else$
\usepackage{etoolbox}
\makeatletter
\providecommand{\subtitle}[1]{% add subtitle to \maketitle
  \apptocmd{\@title}{\par {\large #1 \par}}{}{}
}
\makeatother
$endif$
\subtitle{$subtitle$}
$endif$
\author{$for(author)$$author$$sep$ \and $endfor$}
\date{$date$}
$if(beamer)$
$if(institute)$
\institute{$for(institute)$$institute$$sep$ \and $endfor$}
$endif$
$if(titlegraphic)$
\titlegraphic{\includegraphics$if(titlegraphicoptions)$[$for(titlegraphicoptions)$$titlegraphicoptions$$sep$, $endfor$]$endif${$titlegraphic$}}
$endif$
$if(logo)$
\logo{\includegraphics{$logo$}}
$endif$
$endif$


\begin{document}
$if(has-frontmatter)$
\frontmatter
$endif$
$if(title)$
$if(beamer)$
\frame{\titlepage}
$else$
\maketitle
$endif$
$if(abstract)$
\begin{abstract}
$abstract$
\end{abstract}
$endif$
$endif$

$for(include-before)$
$include-before$

$endfor$
$if(toc)$
$if(toc-title)$
\renewcommand*\contentsname{$toc-title$}
$endif$
$if(beamer)$
\begin{frame}[allowframebreaks]
$if(toc-title)$
  \frametitle{$toc-title$}
$endif$
  \tableofcontents[hideallsubsections]
\end{frame}
$else$
{
$if(colorlinks)$
\hypersetup{linkcolor=$if(toccolor)$$toccolor$$else$$endif$}
$endif$
\setcounter{tocdepth}{$toc-depth$}
\tableofcontents
}
$endif$
$endif$
$if(lof)$
\listoffigures
$endif$
$if(lot)$
\listoftables
$endif$
$if(linestretch)$
\setstretch{$linestretch$}
$endif$
$if(has-frontmatter)$
\mainmatter
$endif$
$body$

$if(has-frontmatter)$
\backmatter
$endif$
$if(natbib)$
$if(bibliography)$
$if(biblio-title)$
$if(has-chapters)$
\renewcommand\bibname{$biblio-title$}
$else$
\renewcommand\refname{$biblio-title$}
$endif$
$endif$
$if(beamer)$
\begin{frame}[allowframebreaks]{$biblio-title$}
  \bibliographytrue
$endif$
  \bibliography{$for(bibliography)$$bibliography$$sep$,$endfor$}
$if(beamer)$
\end{frame}
$endif$

$endif$
$endif$
$if(biblatex)$
$if(beamer)$
\begin{frame}[allowframebreaks]{$biblio-title$}
  \bibliographytrue
  \printbibliography[heading=none]
\end{frame}
$else$
\printbibliography$if(biblio-title)$[title=$biblio-title$]$endif$
$endif$

$endif$
$for(include-after)$
$include-after$

$endfor$
% \maketitle

% \makeatletter
% \newcommand*{\rom}[1]{\expandafter\@slowromancap\romannumeral #1@}
% \makeatother
% \newcommand<>\goal[4][]{\begin{visibleenv}#5\begin{tikzpicture}[node distance=.75ex]
% 	\node[text width=0.43\textwidth,align=left] (H) {\color{accent}\bfseries#3};
% 	\node[text width=0.43\textwidth,align=left,below=of H.south west,anchor=north west] (D) {#4};
% 	\node[roundednode,fit=(H)(D),node on layer=background,shadow,draw=none,fill=gray!20] (F) {};
% 	\node[anchor=north east,on layer=background,opacity=0.15] at (F.north east) {\Huge\color{accent}\bfseries\MakeUppercase{\rom{#2}}};
% 	\if\isempty{#1}\else\begin{scope}[transparency group,opacity=0.7]
% 		\draw[draw=red,lcr,line width=1ex,visible on=<#1>] (F.north west) -- (F.south east);
% 		\draw[draw=red,lcr,line width=1ex,visible on=<#1>] (F.south west) -- (F.north east);
% 	\end{scope}\fi
% \end{tikzpicture}\par\bigskip\end{visibleenv}}
% \begin{frame}[wide]
% 	\frametitle{Learning Objectives}
% 	\begin{wide}
% 	\begin{multicols}{2}
% 		\goal<2->{1}{Why do we need lenses?}{Understand where the idea of lenses come from, and how
% 			one could have come up with them.}
% 		\goal<3->{2}{How can I use them?}{Know the basic functions and operators and know how to
% 			discover new ones.}
% 		\goal<4->{3}{What else is there?}{Know of other lens-like abstractions, why we
% 			presumably need them, and how they differ.}
% 		\goal<5->[6-]{4}{WTF are those types?}{Understand the ins and outs of the lens package and
% 			every type.}
% 	\end{multicols}
% 	\end{wide}
% \end{frame}

% \section{What}

% \colorlet{m}{blue!45!black}
% \tikzset{q/.style={chamfered rectangle,draw,lw,chamfered rectangle xsep=2cm,fill=m,text=white}}
% \tikzset{qa/.style={q,text width=0.4\textwidth}}
% \def\qn#1{\textcolor{orange!80!black}{\textbf{#1}}}
% \begingroup
% \colorlet{accent}{m}
% \makeatletter\smile@listings@initcolors\makeatother
% \setbeamercolor{footline}{fg=white,bg=m}
% \setbeamercolor{footlineright}{fg=white,bg=m}
% \begin{frame}[t,fragile]
% 	\frametitle{What are lenses}
% 	\begin{wide}\vspace{5mm}
% 		\begin{haskell}
% 			type Lens s t a b = forall f. Functor f => (a -> f b) -> s -> f t
% 		\end{haskell}
% 		\begin{visibleenv}<2->
% 		\begin{tikzpicture}[o,node distance=.5ex]
% 			\draw[draw=none,top color=white,bottom color=m] (current page.west) rectangle ([yshift=1.8ex]current page.south east);

% 			\node[q] at ([yshift=-5mm]current page) (Q) {What is the purpose of a lens, according to the types above?};
% 			\coordinate (AS) at ([yshift=-1.3cm]Q.south);

% 			\node[qa,left=of AS,left,muted on=<3-|handout:2->] (A) {\scriptsize\qn{A:} A package for creating visualizations};
% 			\node[qa,right=of AS,right] (B) {\scriptsize\qn{B:} A tool for handling nested ADTs};

% 			\node[qa,below=of A,muted on=<3-|handout:2->] (C) {\scriptsize\qn{C:} A framework for building UIs};
% 			\node[qa,below=of B,muted on=<3-|handout:2->] (D) {\scriptsize\qn{D:} A package for simulating optical lenses};

% 			\draw[lw,short=-1pt,muted on=<3-|handout:2->] (A.east) to (B.west);
% 			\draw[lw,short=-1pt,muted on=<3-|handout:2->] (C.east) to (D.west);

% 			\draw[lw,short=-1pt] (Q.west) to (Q.west -| current page.west);
% 			\draw[lw,short=-1pt] (Q.east) to (Q.east -| current page.east);

% 			\draw[lw,short=-1pt,muted on=<3-|handout:2->] (A.west) to (A.west -| current page.west);
% 			\draw[lw,short=-1pt,muted on=<3-|handout:2->] (C.west) to (C.west -| current page.west);
% 			\draw[lw,short=-1pt,muted on=<3-|handout:2->] (B.east) to (B.east -| current page.east);
% 			\draw[lw,short=-1pt,muted on=<3-|handout:2->] (D.east) to (D.east -| current page.east);
% 		\end{tikzpicture}
% 		\end{visibleenv}
% 	\end{wide}
% 	\begin{modal}<4|handout:3>
% 		\begin{quote}
% 			In Haskell, types provide a pretty good explanation of what a function does. Good luck deciphering lens types.
% 		\end{quote}\par
% 		\hfill Roman Cheplyaka
% 	\end{modal}
% \end{frame}
% \endgroup

% \begin{frame}
% 	\frametitle{What are lenses}
% 	\uncover<2->{Well, \enquote{lens} is also a \link package to https://hackage.haskell.org/package/lens;}
% 	\uncover<3->{\ldots{} Here are some random functions and operators from that package:\bigskip}
% 	\begin{visibleenv}<4->
% 	\begin{center}
% 		\begin{tabular}{lcl}
% 		\hline
% 		\hn{view} & \hn{_1}  & \hn{allOf}    \\ \hline
% 		\hn{set}  & \hn{^.}  & \hn{anyOf}    \\ \hline
% 		\hn{over} & \hn{^?!} & \hn{concatOf} \\ \hline
% 		\end{tabular}
% 	\end{center}\bigskip
% 	\end{visibleenv}
% 	\uncover<5->{We'll shortly see what they do and how we can use them.}
% \end{frame}

% \section{Why}

% \begin{frame}[fragile]
% 	\frametitle{Why do we need them}
% 	\onslide<2->%
% 	Imagine you want to parse configuration files in Haskell. To model them, you come up
% 	with the following ADTs:\par\medskip
% 	\onslide<3->%
% 	\begin{haskell}
% data File = File {
%   name    :: String,
%   entries :: [Entry]
% }
% §\onslide<4->§data Entry = Entry {
%   key   :: String,
%   value :: Value
% }
% §\onslide<5->§data Value = Value {
%   curr :: String,
%   def  :: String
% }
% 	\end{haskell}
% \end{frame}

% \begin{frame}[fragile]
% 	\frametitle{Let's reinvent the lens}
% 	\onslide<2->Now, we can build our lens abstraction:
% 	\begin{xshaskell}
% 		§\onslide<3->§data Lens s a = Lens {
% 			§\onslide<4->§get :: s -> a,
% 			§\onslide<5->§modify :: (a -> a) -> s -> s
% 		§\onslide<3->§}
% 	\end{xshaskell}\medskip

% 	\onslide<6->We need to reimplement the function composition:
% 	\begin{xshaskell}
% 		§\onslide<7->§compose :: Lens a b -> Lens b c -> Lens a c
% 		§\onslide<8->§compose (Lens g m) (Lens g' m') = Lens {
% 			§\onslide<9->§get = g' . g,
% 			§\onslide<10->§modify = m . m'
% 		§\onslide<8->§}
% 	\end{xshaskell}\medskip

% 	\onslide<11->For easier handling, we also define \h{set} as a little helper:
% 	\begin{xshaskell}
% 		§\onslide<12->§set :: Lens s a -> a -> s -> s
% 		§\onslide<13->§set (Lens _ modify) = modify§\onslide<14->§ . const
% 	\end{xshaskell}
% \end{frame}

% \begin{frame}[fragile]
% 	\frametitle{Let's reinvent the lens}
% 	\onslide<2->Our solution looks more flexible than what we had before. But there are still some
% 	problems:\bigskip
% 	\begin{itemize}[<+(2)->]
% 		\item Still feels a bit clunky and boilerplate-heavy
% 		\item We always have to create \h{Lens} values
% 		\item No support for polymorphic updates
% 	\end{itemize}\bigskip
% 	\onslide<8->It's definitely not impossible to overcome these limitations, but we'll skip this
% 	for now.
% 	\onslide<1->\begin{modal}<6|handout:2>[\footnotesize Polymorphic Update]
% 		\begin{haskell}
% 			data Pair a b = Pair {e1 :: a, e2 :: b}

% 			p :: Pair Int String
% 			p = Pair 420 "is fun"

% 			p { e1 = "FP" }  §\ergo\texttt{~~~Pair \{ e1 = "FP", e2 = "is fun" \}}§
% 		\end{haskell}\bigskip
% 		\ergo{} Notice that the type has changed from \h{Pair Int String} to \h{Pair
% 		String String}. This is what we call \emph{polymorphic update}.
% 	\end{modal}
% \end{frame}

% \begin{frame}
% 	\frametitle[Revisited]{What are lenses}
% 	\onslide<2->Lenses are:
% 	\begin{itemize}[<+(2)->]
% 		\item A way to \emph{focus} on a part of a data structure
% 	\end{itemize}\bigskip
% 	\onslide<4->Or more precisely:
% 	\begin{itemize}[<+(3)->]
% 		\item Just another abstraction
% 		\item Functional references
% 		\item Getters and Setters
% 		\item Highly composable  and flexible
% 			\begin{itemize}
% 				\item \enquote{The Power is in the Dot}\quad\textcolor{gray}{Edward Kmett}
% 			\end{itemize}
% 	\end{itemize}
% \end{frame}

% \begin{frame}
% 	\frametitle[\cite{lenshist}]{A little history lesson}
% 	\begin{itemize}[<+(1)->]\setlength\itemsep{.7em}
% 		% \item[\tikzmark{a}] Jeremy Gibbons and Bruno Oliveira demonstrate that
% 		% 	Traversals encode the Iterator pattern~\cite{gibbons2009}
% 		\item[\tikzmark{b}] Luke Palmer creates a pattern he calls \emph{Accessors} to
% 			ease stateful programming in Haskell~\cite{palmer2007game}. He uses C's
% 			preprocessor to generate \h{readVal} and \h{writeVal} functions.\footnote{In
% 			another blog post he then swaps out the preprocessor in favour of Template
% 			Haskell.}
% 		\item[\tikzmark{c}] Palmer generalizes his Accessors into something more like
% 			today's lenses.~\cite{palmer2007acc}
% 		\item[\tikzmark{d}] Twan van Laarhoven comes up with a novel way to express
% 			lenses using the \h{Functor} class~\cite{laarhoven2009}. We call them
% 			\emph{van Laarhoven lenses}.
% 	\end{itemize}
% 	\begin{tikzpicture}[remember picture,overlay]
% 		\coordinate (S) at ([shift={(-2mm,1.5mm)}]pic cs:b);
% 		\coordinate (E) at ([shift={(-2mm,0cm)}]pic cs:d);
% 		\draw[lw] (S) -- (E);
% 		\draw[dashed,lw] (E) -- ([yshift=-1cm]E);
% 		\node[roundnode,fill=accent,draw=accent,inner sep=1.5pt,visible on=<2->] at (S) {};
% 		% \node[roundnode,fill=accent,draw=accent,inner sep=1.5pt,visible on=<5->] at ([shift={(-2mm,1.5mm)}]pic cs:b) {};
% 		\node[roundnode,fill=accent,draw=accent,inner sep=1.5pt,visible on=<3->] at ([shift={(-2mm,1.5mm)}]pic cs:c) {};
% 		\node[roundnode,fill=accent,draw=accent,inner sep=1.5pt,visible on=<4->] at ([shift={(-2mm,1.5mm)}]pic cs:d) {};
% 	\end{tikzpicture}
% \end{frame}

% \begin{frame}
% 	\frametitle[\cite{lenshist}]{A little history lesson}
% 	\begin{itemize}[<+(1)->]\setlength\itemsep{.7em}
% 		\item[\tikzmark{e}] Russell O'Connor realises van Laarhoven lenses have always
% 			supported polymorphic updates.~\cite{connor2012}
% 		\item[\tikzmark{f}] Edward Kmett realises that you can put laws on the notion of
% 			polymorphic updates.~\cite{kmett2012}
% 		\item[\tikzmark{g}] Kmett pushed the \link first commit to
% 			https://github.com/ekmett/lens/commit/c5c8e5ffeeccdd7ac78f758dfc5723c411443d78;
% 			to the lens repository on \link GitHub to https://github.com/ekmett/lens;
% 	\end{itemize}
% 	\begin{tikzpicture}[remember picture,overlay]
% 		\coordinate (S) at ([shift={(-2mm,1.5mm)}]pic cs:e);
% 		\coordinate (E) at ([shift={(-2mm,-1.1cm)}]pic cs:g);
% 		\draw[dashed,lw] ([yshift=7.5mm]S) -- (S);
% 		\draw[-{Latex[round,accent]},lw] (S) -- (E);
% 		\node[roundnode,fill=accent,draw=accent,inner sep=1.5pt,visible on=<2->] at (S) {};
% 		\node[roundnode,fill=accent,draw=accent,inner sep=1.5pt,visible on=<3->] at ([shift={(-2mm,1.5mm)}]pic cs:f) {};
% 		\node[roundnode,fill=accent,draw=accent,inner sep=1.5pt,visible on=<4->] at ([shift={(-2mm,1.5mm)}]pic cs:g) {};
% 	\end{tikzpicture}
% \end{frame}

% \section{How}
% \subsection{A little Overview}

% \begin{frame}
% 	\frametitle{A little Overview}
% 	\onslide<2->Lenses basically provide two kinds of operations:
% 	\begin{itemize}[<+(2)->]
% 		\item \hn{view :: Lens' s a -> s -> a}
% 		\item \hn{set :: Lens' s a -> a -> s -> s}
% 	\end{itemize}\medskip\onslide<5->
% 	To use them, we need the actual lens. It determines what part of the structure
% 	we want to focus on.
% 	\begin{itemize}[<+(3)->]
% 		\item \hn{_1 :: Lens' (a,b) a}
% 		\item \hn{_2 :: Lens' (a,b) b}
% 	\end{itemize}\medskip\onslide<8->
% 	With all that in place, we can now combine the operation with a lens (or a
% 	combination of lenses) and data:
% 	\begin{itemize}
% 		\item<9-> \hn{set _2 "cool" ("FP is", "")}\onslide<10->\kern8pt\ergo\kern8pt\hn{("FP is", "cool")}
% 		\item<11-> \hn{view _1 ("hi", "there")}\onslide<12->\kern8pt\ergo\kern8pt\hn{"hi"}
% 		% \item \doul{red}{\hn{set}} \doul{blue}{\hn{_2}} \doul{green}{\hn{"cool"}} \doul{green}{\hn{("FP is", "")}}\kern8pt\ergo\kern8pt\hn{("FP is", "cool")}
% 		% \item \doul{red}{\hn{view}} \doul{blue}{\hn{_1}} \doul{green}{\hn{("hi", "there")}}\kern8pt\ergo\kern8pt\hn{"hi"}
% 	\end{itemize}%
% \end{frame}

% \subsection{Lens Laws}

% \begin{frame}
% 	\frametitle{Lens Laws}
% 	\onslide<2->Like with functors, applicatives, and monads, lenses \emph{should} follow some
% 	rules:
% 	\begin{enumerate}[<+(2)->]
% 		\item Get-Put
% 		\item Put-Get
% 		\item[\emoji{hatching-chick}] Put-Put
% 	\end{enumerate}
% 	\onslide<6->We'll look at them in a bit more detail.
% \end{frame}

% \begin{frame}[fragile]
% 	\frametitle[Get-Put]{Lens Laws}
% 	\onslide<2->If you modify something by changing its subpart to exactly what it was before,
% 	nothing should happen.\bigskip
% 	\onslide<3->\begin{haskell}
% 		set entryValueL (get entryValueL entry) entry == entry
% 	\end{haskell}\bigskip
% 	\onslide<4->\ergo{} The lens should not modify the value or structure by itself.
% \end{frame}

% \begin{frame}[fragile]
% 	\frametitle[Put-Get]{Lens Laws}
% 	\onslide<2->If you modify something by inserting a particular subpart and then view the result,
% 	you'll get back exactly that subpart.\bigskip
% 	\onslide<3->\begin{haskell}
% 		get entryValueL (set entryValueL v entry) == v
% 	\end{haskell}\bigskip
% 	\onslide<4->\ergo{} Setting values should be independent of any previous state.
% \end{frame}

% \begin{frame}[fragile]
% 	\frametitle[Put-Put]{Lens Laws}
% 	\onslide<2->If you modify something by inserting a particular subpart \h{a}, and then modify it
% 	again inserting a different subpart \h{b}, it's exactly as if you only did the
% 	second insertion.\bigskip
% 	\onslide<3->\begin{haskell}
% 		set entryValueL v2 (set entryValueL v1 entry) == set entryValueL v2 entry = 1
% 	\end{haskell}\bigskip
% 	\onslide<4->\ergo{} Previous updates should not leave any traces.
% \end{frame}

% \begin{frame}
% 	\frametitle{Do I really have to follow them?}
% 	\begin{itemize}[<+(1)->]
% 		\item Yes, you should! Otherwise your lenses might behave weird.
% 		\item And weird unpredictable things are for OOP \emoji{wink}
% 	\end{itemize}\bigskip
% 	\begin{itemize}[<+(1)->]
% 		\item But, we can get around them
% 		\item In fact, we can get around the whole process of creating a lens by
% 			hand
% 		\item You remember Template-Haskell, do you?
% 	\end{itemize}
% \end{frame}

% \begin{frame}[fragile]
% 	\frametitle{Do I really have to follow them?}
% 	\begin{haskell}
% 		§\onslide<2->§{-# LANGUAGE TemplateHaskell #-}

% 		§\onslide<3->§import Control.Lens

% 		§\onslide<4->§data File  = File  {_name :: String, _entries :: [Entry]}
% 		§\onslide<4->§data Entry = Entry {_key  :: String, _value   :: Value  }
% 		§\onslide<4->§data Value = Value {_curr :: String, _def     :: String }

% 		§\onslide<5->§makeLenses ''File
% 		§\onslide<5->§makeLenses ''Entry
% 		§\onslide<5->§makeLenses ''Value
% 	\end{haskell}
% \end{frame}

% \subsection{The actual Package}

% \begin{frame}
% 	\frametitle{The lens Package}
% 	\begin{itemize}[<+(1)->]
% 		\item Until now, we have only used \h{view} and \h{set}
% 		\item But there are actually a lot more functions and operators
% 		\item I mean a loooooooooooooooooot; easily over 100
% 	\end{itemize}\bigskip
% 	\begin{itemize}
% 		\item<5-> Let's try to find a pattern in their names
% 	\end{itemize}
% \end{frame}

% \begin{frame}
% 	\frametitle[Getters]{The lens Package}
% 	\onslide<2->Writing a \emph{Getter} is really easy. We can simply promote any \emph{function} or
% 	\emph{value} to a Getter.\par\bigskip
% 	\begin{itemize}[<+(2)->]\setlength\itemsep{10pt}
% 		\item \h{to} builds a Getter from any function\par\medskip
% 			\hn{("Hello", "FP2") ^. to snd}\quad\ergo\quad\hn{"FP2"}
% 		\item \h{like} always returns a constant value\par\medskip
% 			\hn{("Hello", "FP2") ^. like 42}\quad\ergo\quad\hn{42}
% 	\end{itemize}
% \end{frame}

% \begin{frame}
% 	\frametitle[Setters]{The lens Package}
% 	\onslide<2->Writing a \emph{Setter} is only slightly more complicated, as we don't set the value
% 	directly, but apply a function on the focused part.\par\bigskip
% 	\begin{itemize}[<+(2)->]\setlength\itemsep{10pt}
% 		\item \h{setting} receives a function, that applies another function to the
% 			correct value inside a structure\par\medskip
% 			\hn{(4,1) & setting (\\f (x,y) -> (x,f y)) .~ 2}\quad\ergo\quad\hn{(4,2)}
% 		\item \h{sets} is in theory a bit more flexible, but that's out of scope for
% 			today\par\medskip
% 			\hn{(4,1) & sets (\\f (x,y) -> (x,f y)) .~ 2}\quad\ergo\quad\hn{(4,2)}
% 	\end{itemize}
% \end{frame}

% \begin{frame}[fragile]
% 	\frametitle[Getter + Setter]{The lens Package}
% 	\onslide<2->Having a separate Getter and Setter is not always desirable. Now, we want to create
% 	our own lens that we can use as both Getter and Setter. This time, \h{makeLenses}
% 	doesn't count!\par\bigskip
% 	\begin{itemize}\setlength\itemsep{10pt}
% 		\item<3-> We can use \h{lens} to combine a viewing and setting function
% 			\begin{haskell}
% 				§\onslide<4->§g = snd
% 				§\onslide<5->§s = (\(a,_) b -> (a,b))
% 				§\onslide<6->§_2 = lens g s
% 			\end{haskell}
% 		\item<7-> You can also simply write a custom function with the type \h{l :: forall
% 			f. Functor f => (a -> f b) -> s -> f t} that satisfies all three lens laws.
% 			Good luck! \onslide<8->We'll try it anyway.
% 	\end{itemize}
% \end{frame}

% \section{More Goodies}

% \subsection{Virtual lenses}
% \begin{frame}[fragile]
% 	\frametitle{Virtual lenses}
% 	\onslide<2->A Getter does not always have to be backed by an actual structure. Theoretically, it
% 	can return \emph{anything}:
% 	\onslide<3->\begin{lstlisting}[language=ts]
% 		get virtualProp(): number {
% 			return 42
% 		}
% 	\end{lstlisting}\bigskip
% 	\onslide<4->We can easily achieve this behavior with lenses, too:
% 	\begin{haskell}
% 		§\onslide<5->§virtualProp = like 42
% 		§\onslide<8->§(0,0) §\onslide<6->§^.§\onslide<7->§ virtualProp §\onslide<9->\ergo~\texttt{42}§
% 	\end{haskell}
% \end{frame}

% \subsection{Prisms}

% \begin{frame}[fragile]
% 	\frametitle{Prisms}
% 	% \begin{sblock}[]
% 	% 	Lenses reference something that \emph{always} exists.
% 	% 	\tcblower
% 	% 	Prisms reference something that \emph{may} exist.
% 	% \end{sblock}\bigskip
% 	\onslide<2->So far, we only looked at product types. But what about sum types? Prisms to the
% 	rescue!\par\bigskip
% 	\begin{haskell}
% §\onslide<3->§data CanteenMeal = MainCourse String CanteenMeal
%                  | Desert String

% §\onslide<4->§meal1 = MainCourse "Sattmacher" (Desert "Pudding")
% §\onslide<5->§meal2 = Desert "Yogurt"

% §\onslide<8->§meal1 §\onslide<6->§^?§\onslide<7->§ _MainCourse . _2 . _Dessert §\onslide<9->\ergo~\texttt{Just "Pudding"}§
% §\onslide<10->§meal2 ^? _MainCourse . _2 . _Dessert §\onslide<11->\ergo~\texttt{Nothing}§

% §\onslide<15->§meal1 &  §\onslide<13->§_MainCourse . _2 . _Dessert §\onslide<12->§.~ §\onslide<14->§"Yogurt"
% §\onslide<16->\ergo~\texttt{Desert "Yogurt" \color{gray}inside meal1}§
% 	\end{haskell}
% \end{frame}

% \subsection{Traversals}
% \newsavebox\wwm
% \tikzset{modal box/.append style={align=center,fill=m!20}}
% \begin{frame}[fragile]
% 	\frametitle{Traversals}
% 	% \begin{columns}[c]
% 	% 	\begin{column}{0.45\textwidth}
% 	% 		\begin{block}
% 	% 		Lenses reference something that \emph{always} exists.
% 	% 		\end{block}
% 	% 	\end{column}
% 	% 	\begin{column}{0.45\textwidth}
% 	% 		\begin{block}
% 	% 		Traversals reference \emph{many things} that \emph{may} exist.
% 	% 		\end{block}
% 	% 	\end{column}
% 	% \end{columns}\bigskip
% 	\savebox\wwm{\begin{tikzpicture}[node distance=.5ex]
% 		% \draw[draw=none,top color=white,bottom color=m] (current page.west) rectangle ([yshift=1.8ex]current page.south east);

% 		\node[q] at ([yshift=-5mm]current page) (Q) {What do you think will \texttt{["a","b"] \textasciicircum. traverse} return?};
% 		\coordinate (AS) at ([yshift=-1.3cm]Q.south);

% 		\node[qa,left=of AS,left] (A) {\scriptsize\qn{A:} \texttt{["a","b"]}};
% 		\node[qa,right=of AS,right] (B) {\scriptsize\qn{B:} \texttt{"a"}};

% 		\node[qa,below=of A] (C) {\scriptsize\qn{C:} \texttt{"ab"}};
% 		\node[qa,below=of B] (D) {\scriptsize\qn{D:} \texttt{("a", "b")}};

% 		\draw[lw,short=-1pt] (A.east) to (B.west);
% 		\draw[lw,short=-1pt] (C.east) to (D.west);
% 	\end{tikzpicture}}
% 	\onslide<2->Wouldn't it be nice to have a lens that focuses on a specific element of
% 	a traversable container? Let's start with every element:
% 	\begin{haskell}
% 		§\onslide<5->§["Hello", "there"] §\onslide<3->§^.§\onslide<4->§ traverse §\onslide<8->\ergo~\texttt{"Hellothere"}§
% 	\end{haskell}\bigskip
% 	\onslide<9->Huh?! What's that? I would've expected \texttt{["Hello", "there"]}.\par
% 	\onslide<10->When viewing the result of \h{traverse}, it gets shoved through
% 	\h{mappend} first. \onslide<11->That's why you typically \h{^..}.
% 	% allOf (traverse . _2) even [(1,2),(3,4)] §\ergo~\texttt{True}§
% 	\begin{haskell}
% 		§\onslide<14->§[1..5] §\onslide<12->§^..§\onslide<13->§ traverse §\onslide<15->\ergo~\texttt{[1,2,3,4,5]}§
% 		§\onslide<16->§[(1,2),(3,4)] ^.. traverse . _2 §\onslide<17->\ergo~\texttt{[2,4]}§

% 		§\onslide<21->§[1..5] & §\onslide<19->§traverse §\onslide<18->§+~§\onslide<20->§ 1 §\onslide<22->\ergo~\texttt{[2,3,4,5,6]}§
% 	\end{haskell}\onslide<1->
% 	\mode<beamer>{\begin{modal}<6>
% 		\scalebox{0.9}{\usebox\wwm}
% 	\end{modal}}
% 	% \begin{modal}<2|handout:2>[Folds]
% 	% 	\begin{center}
% 	% 	\begin{tabular}{lll}
% 	% 		\hn{allOf} & \hn{anyOf} & \hn{noneOf}\\
% 	% 		\hn{sumOf} & \hn{productOf} & \hn{lengthOf}\\
% 	% 		\hn{concatOf} & \hn{elemOf} & \hn{notElemOf}\\
% 	% 		\hn{maximumOf} & \hn{minimumOf} & \hn{findOf}
% 	% 	\end{tabular}
% 	% 	\end{center}\medskip
% 	% 	You see the pattern, just append \hn{Of} to the function name, and tadaa, here's
% 	% 	your new fold that works with lenses.
% 	% \end{modal}
% \end{frame}

% \begin{frame}[fragile]
% 	\frametitle{Traversals}
% 	As promised, here's how we can focus on a specific element of a traversable:
% 	\begin{haskell}
% 		[1..5] ^.. ix 1 §\ergo~\texttt{[2]}§
% 		[1..5] ^.. ix 5 §\ergo~\texttt{[]}§
% 	\end{haskell}\bigskip
% 	Returning an empty list on failure does not seem very nice. Let's use the
% 	prism-view-operator to get a \h{Maybe}:\par
% 	\begin{haskell}
% 		[1..5] ^? ix 1 §\ergo~\texttt{Just 2}§
% 		[1..5] ^? ix 5 §\ergo~\texttt{Nothing}§
% 	\end{haskell}
% \end{frame}

% \subsection{Isos}
% \begin{frame}
% 	\frametitle{Isos}
% 	\onslide<2->Here's a very short summary:
% 	\begin{itemize}[<+(2)->]
% 		\item An \h{Iso} is a connection between two types that are equivalent in every way
% 		\item Isos should follow the following laws:\par
% 			\hn{forward . backward = id}\par
% 			\hn{backward . forward = id}
% 		\item We can write our own \h{Iso} by providing a forward and backward mapping
% 	\end{itemize}
% \end{frame}

% \begin{frame}[fragile]
% 	\frametitle{Isos}
% 	\begin{haskell}
% 		§\onslide<2->§maybeToEither = maybe (Left ()) Right
% 		§\onslide<3->§eitherToMaybe = either (const Nothing) Just

% 		§\onslide<4->§someIso :: Iso' (Maybe a) (Either () a)
% 		§\onslide<5->§someIso = iso §\onslide<6->§maybeToEither §\onslide<7->§eitherToMaybe

% 		§\onslide<10->§Just "hi" §\onslide<8->§^. §\onslide<9->§someIso §\onslide<11->\ergo~\texttt{Right "hi"}§
% 		§\onslide<12->§Left "ho" ^. from someIso §\onslide<13->\ergo~\texttt{Nothing}§
% 	\end{haskell}
% \end{frame}

% \section{Summary}
% \tikzset{
% 	u/.style={roundednode,draw=none,shadow,fill=gray!20},
% 	% u/.style={roundednode,fill=gray!20},
% 	v/.style={arrows={-Stealth[round,inset=0pt,length=10pt,angle'=90,open]},lw,lcr,rnd},
% }
% \def\card#1#2{\begin{tikzpicture}[node distance=0]
% 	\node[text width=0.45\textwidth] (C) {#2};
% 	\node[above=of C,yshift=-2mm] (T) {#1};
% 	\node[u,fit=(T)(C),node on layer=background] () {};
% \end{tikzpicture}}

% \definecolor{dblue}{HTML}{000742}
% \newsavebox\duckforscale\newsavebox\sunb
% \savebox\duckforscale{\tikz\duck[peakedcap=dblue];}
% \makeatletter
% \savebox\sunb{\begin{tikzpicture}
% 	\filldraw[draw=orange,fill=yellow,line width=6.6\smile@linewidth] (5,0) arc (0:360:5);
% 	\foreach\angle in {0,45,...,360}{
% 		\filldraw[draw=orange,fill=yellow,line width=5\smile@linewidth,rotate around={\angle:(0,0)}]
% 			(5.5,0) -- +(0,-1.5) -- +(3,0) -- +(0,1.5) -- cycle;
% 	}
% \end{tikzpicture}}
% \begingroup
% \colorlet{accent}{dblue}
% \setbeamercolor{footline}{fg=white,bg=dblue}
% \setbeamercolor{footlineright}{bg=yellow}
% \begin{frame}
% 	\frametitle{And so much more}
% 	\begin{wide}\centering
% 	\begin{tikzpicture}[o,every node/.append style={execute at begin node=\scriptsize}]
% 		\fill[shade,top color=white,bottom color=dblue] (current page.west) rectangle (current page.south east);
% 		\filldraw[decoration={random steps,segment length=3mm,amplitude=1mm},decorate,
% 			rnd,lw,draw=dblue,
% 			shade,top color=white,bottom color=dblue!50,middle color=dblue!15
% 		] (0,-3.5) -- (-3,0) -- (0,2) -- (3,0) -- (0,-3.5);
% 		\draw[dblue,decorate,decoration={snake,amplitude=1mm,segment length=2cm},lw] let \p1=(current page.west) in (\x1,0) -- (-3,0);
% 		\draw[dblue,decorate,decoration={snake,amplitude=1mm,segment length=2cm},lw] let \p1=(current page.east) in (3,0) -- (\x1,0);

% 		\node at (4.5,0.07) {\rotatebox{-15}{\scalebox{.15}{\usebox\duckforscale}}};
% 		\node[anchor=north east] at (current page.north east) {\scalebox{.15}{\usebox\sunb}};

% 		\node at (0,1.2) {General idea};
% 		\node at (1,0.5) {Lens laws};
% 		\node[text width=2cm,align=center] at (-0.9,0.5) {Composable lenses};
% 		\node at (0.3,0) {\hn{Lens}};
% 		\node at (-1.4,-0.2) {\hn{Prism}};
% 		\node[text width=11ex,align=center] at (1.5,-0.5) {\hn{Traversal}};
% 		\node at (0,-0.6) {\hn{Fold}};
% 		\node at (-1,-0.65) {\hn{Iso}};
% 		\node[text width=1.7cm,align=center] at (-0.9,-1.2) {Index\\preserving};
% 		\node at (0.8,-1) {\hn{Bazaar}};
% 		\node at (0.4,-1.7) {Subtyping};
% 		\node[text width=9ex,align=center] at (0,-2.3) {Typesignatures};
% 	\end{tikzpicture}
% 	\end{wide}
% \end{frame}
% \endgroup

% \section{References}
% \defbibheading{bibliography}[\bibname]{}

% \begin{frame}[allowframebreaks]
% 	\frametitle{Reading suggestions}
% 	\printbibliography[keyword={suggestion}]
% \end{frame}

% \begin{frame}[allowframebreaks]
% 	\frametitle{References}
% 	\printbibliography
% \end{frame}
\end{document}
